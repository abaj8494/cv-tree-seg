\section{Literature Review}
Dead trees are defined as the deadwood and the dead parts, even the living trees which have the dead part \cite{b2}. The dead trees are vital to the entire forest ecosystem, which plays a positive role in providing a habitat for fungal plants and animals while preventing soil erosion \cite{b2}. However, an increasing number of trees are drying up or even dying due to extreme droughts and heat waves caused by global warming \cite{b3}. This poses significant threats to both ecological balance and public safety, as evidenced by incidents where standing dead trees fell on roads causing accidents. In 2015, there were two accidents about hitting passing vehicles because of standing dead trees, which caused serious damage to people’s health and property \cite{b29}. To prevent similar incidents from happening again, the government has formulated relevant laws. Experts and researchers have begun to study computer vision technology to detect and manage tree health. Initially, forest inventory was done by manual field measurements. It needs a lot of investigators and time to search and confirm, and the cost is very expensive \cite{b4}. To address these limitations, remote sensing technology emerged as a more efficient alternative, enabling larger scale forest monitoring through spatial data acquisition \cite{b1}. Remote sensing technology can provide continuous spatial information and reliable data to help people understand the location, quantity and dynamics of standing dead trees \cite{b5}. It is the cheapest and simplest method to use and draw the maps \cite{b2}. Recent studies have explored various data acquisition methods including satellite imagery, lidar sensors, unmanned aerial vehicles, each offering unique advantages in terms of coverage, resolution and cost-effectiveness \cite{b6}. In the field of computer vision, researchers have developed numerous approaches for dead tree detection and segmentation. These range from traditional image processing techniques to more advanced machine learning and deep learning methods. Particularly noteworthy are convolutional neural networks (CNN) and their variants, which have demonstrated superior performance in dead tree segmentation tasks \cite{b6}. Other effective approaches include random forests (RF) and support vector machines (SVM), which offer different trade-offs between accuracy and computational efficiency \cite{b7}. The integration of multi-spectral data analysis with these computational methods has opened new possibilities for improving segmentation accuracy and reliability in diverse forest environments \cite{b6}. 